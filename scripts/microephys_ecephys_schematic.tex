\documentclass[border=10pt]{standalone}
\usepackage{tikz}
\usepackage{xcolor}
\usepackage{booktabs}
\usepackage{dirtree}
\usepackage{array}
\usepackage{graphicx}
\usepackage{helvet}
\renewcommand{\familydefault}{\sfdefault}
\usetikzlibrary{positioning,arrows.meta,shapes,backgrounds,fit,calc}

% Color definitions
\definecolor{treebg}{HTML}{F5F5F5}
\definecolor{boxbg}{HTML}{FFFFFF}
\definecolor{boxborder}{HTML}{CCCCCC}
\definecolor{highlight}{HTML}{FFF9E6}
\definecolor{darktext}{HTML}{2C3E50}
\definecolor{arrowcol}{HTML}{3498DB}

\newcommand{\filelabel}[1]{%
  \tikz[]\node[draw=arrowcol,inner sep=0pt,minimum size=0.45cm,font=\bfseries]{#1};%
}

\begin{document}
\begin{tikzpicture}[
    node distance=0.4cm,
    contentbox/.style={
        rectangle,
        draw=boxborder,
        fill=boxbg,
        line width=1pt,
        rounded corners=3pt,
        inner sep=10pt,
        text width=8cm,
        align=left,
        font=\small
    },
    widecontentbox/.style={
        rectangle,
        draw=boxborder,
        fill=boxbg,
        line width=1pt,
        rounded corners=3pt,
        inner sep=10pt,
        text width=10cm,
        align=left,
        font=\small
    },
    narrowcontentbox/.style={
        rectangle,
        draw=boxborder,
        fill=boxbg,
        line width=1pt,
        rounded corners=3pt,
        inner sep=10pt,
        text width=6cm,
        align=left,
        font=\small
    },
    labelstyle/.style={
        % circle,
        draw=arrowcol,
        fill=arrowcol,
        inner sep=0pt,
        minimum size=0.4cm,
        font=\bfseries\large,
        text=white
    }
]

% Title at top
\node[font=\LARGE\bfseries, text=darktext] (title) {
    Extracellular Electrophysiology (ecephys) BIDS BEP032 Dataset
};

% File tree
\node[below=0.8cm of title.south west, anchor=north west, fill=treebg, draw=boxborder, line width=1.5pt, rounded corners=4pt, text width=11.2cm] (treenode) {
{\small\ttfamily
\dirtree{%
.1 dataset/.
.2 README.
.2 CHANGES.
.2 dataset\_description.json.
.2 participants.tsv.
.2 probes/.
.3 custom\_probe.json.
.2 sub-01/.
.3 sub-01\_sessions.tsv.
.3 sub-01\_acq-photo1\_photo.png.
.3 ses-001/.
.4 sub-01\_ses-001\_scans.tsv.
.4 ecephys/.
.5 sub-01\_ses-001\_task-discrimination\_ecephys.nix \filelabel{1}.
.5 sub-01\_ses-001\_task-discrimination\_ecephys.json \filelabel{2}.
%.5 sub-01\_ses-001\_task-discrimination\_events.tsv.
.5 sub-01\_ses-001\_task-discrimination\_events.json.
.5 sub-01\_ses-001\_probes.tsv \filelabel{3}.
.5 sub-01\_ses-001\_channels.tsv \filelabel{4}.
.5 sub-01\_ses-001\_electrodes.tsv \filelabel{5}.
.5 sub-01\_ses-001\_electrodes.json.
.5 sub-01\_ses-001\_space-StereoTaxic\_electrodes.tsv \filelabel{6}.
.5 sub-01\_ses-001\_space-StereoTaxic\_coordsystem.json \filelabel{7}.
.5 sub-01\_ses-001\_probes.json \filelabel{8}.
.5 sub-01\_ses-001\_channels.json.
.3 ses-002/.
.4 ecephys/.
.5 sub-01\_ses-002\_task-reaching\_ecephys.nix.
.5 sub-01\_ses-002\_task-reaching\_ecephys.json.
.5 sub-01\_ses-002\_task-reaching\_events.tsv.
.5 ...
}
}
};

% COLUMN 1 - ecephys files | COLUMN 2 - TSVs | COLUMN 3 - JSONs
% Column 1, Row 1: ecephys.json
\node[narrowcontentbox, right=0.2cm of treenode.north east, anchor=north west] (box4) {
    \textbf{\normalsize\ttfamily *\_ecephys.json}\\[2pt]
    {\footnotesize\textcolor{gray}{Recording Metadata}}\\[4pt]
    {\tiny\ttfamily
    \{\\
    ~~"TaskName": "discrimination",\\
    ~~"TaskDescription": "Two-alternative forced choice...",\\
    ~~"SamplingFrequency": 30000,\\
    ~~"PowerLineFrequency": 60,\\
    ~~"SoftwareFilters": "n/a",\\
    ~~"HardwareFilters": \{\\
    ~~~~"HighpassFilter": \{"Half-amplitude cutoff (Hz)": 0.1\}\\
    ~~\},\\
    ~~"Manufacturer": "ManufacturerName",\\
    ~~"ManufacturersModelName": "AcquisitionSystemModel",\\
    ~~"SoftwareName": "RecordingSoftware",\\
    ~~"SoftwareVersions": "1.0.0",\\
    ~~"PharmaceuticalName": ["AnestheticAgent"],\\
    ~~"BodyPart": "BRAIN",\\
    ~~"BodyPartDetails": "TargetBrainRegion",\\
    ~~"SampleEnvironment": "in-vivo"\\
    \}
    }
};
\node[labelstyle] at ([xshift=-0.4cm, yshift=-0.4cm]box4.north east) {2};

% Column 1, Row 2: ecephys.nix
\node[narrowcontentbox, below=0.6cm of box4.south west, anchor=north west] (box8) {
    \textbf{\normalsize\ttfamily *\_ecephys.nix}\\[2pt]
    {\footnotesize\textcolor{gray}{Electrophysiology Data}}\\[4pt]
    \includegraphics[width=6cm]{../figures/traces.png}
};
\node[labelstyle] at ([xshift=-0.4cm, yshift=-0.4cm]box8.north east) {1};

% Column 2, Row 1: probes.tsv
\node[widecontentbox, right=0.2cm of box4.north east, anchor=north west] (box1) {
    \textbf{\normalsize\ttfamily *\_probes.tsv}\\[2pt]
    {\footnotesize\textcolor{gray}{Probe Specifications}}\\[4pt]
    {\tiny\ttfamily
    \begin{tabular}{@{}llllllll@{}}
    \toprule
    probe\_name & type & AP & ML & DV & AP\_angle & hemisphere & electrode\_count \\
    \midrule
    probe01 & silicon-probe & -2.5 & 1.5 & -4.0 & 15 & L & 384 \\
    probe02 & tetrode & -1.2 & -2.1 & -3.5 & 0 & R & 4 \\
    \bottomrule
    \end{tabular}
    }
};
\node[labelstyle] at ([xshift=-0.4cm, yshift=-0.4cm]box1.north east) {3};

% Column 2, Row 2: channels.tsv
\node[widecontentbox, below=0.6cm of box1.south west, anchor=north west] (box3) {
    \textbf{\normalsize\ttfamily *\_channels.tsv}\\[2pt]
    {\footnotesize\textcolor{gray}{Channel Properties}}\\[4pt]
    {\tiny\ttfamily
    \begin{tabular}{@{}llllllll@{}}
    \toprule
    name & reference & type & units & sampling\_frequency & hardware\_filters & gain & status \\
    \midrule
    ch001 & ref01 & HP & uV & 30000 & HighpassFilter & 500 & good \\
    ch002 & ref01 & HP & uV & 30000 & HighpassFilter & 500 & good \\
    ch003 & ref01 & LFP & uV & 1000 & HighpassFilter & 500 & bad \\
    sync01 & n/a & SYNC & V & 30000 & n/a & 1 & good \\
    \bottomrule
    \end{tabular}
    }
};
\node[labelstyle] at ([xshift=-0.4cm, yshift=-0.4cm]box3.north east) {4};

% Column 2, Row 3: electrodes.tsv
\node[widecontentbox, below=0.6cm of box3.south west, anchor=north west] (box5) {
    \textbf{\normalsize\ttfamily *\_electrodes.tsv}\\[2pt]
    {\footnotesize\textcolor{gray}{Electrodes (probe-relative)}}\\[4pt]
    {\tiny\ttfamily
    \begin{tabular}{@{}llllllllll@{}}
    \toprule
    name & probe\_name & hemisphere & x & y & z & impedance & shank\_id & material & location \\
    \midrule
    e001 & probe01 & L & 0 & 0 & 0 & 1.2 & 0 & iridium-oxide & RegionA \\
    e002 & probe01 & L & 0 & 0 & 25 & 1.1 & 0 & iridium-oxide & RegionA \\
    e003 & probe01 & L & 0 & 0 & 50 & 1.3 & 0 & iridium-oxide & RegionA \\
    e004 & probe01 & L & 0 & 0 & 75 & 1.4 & 0 & iridium-oxide & RegionA \\
    \bottomrule
    \end{tabular}
    }
};
\node[labelstyle] at ([xshift=-0.4cm, yshift=-0.4cm]box5.north east) {5};

% Column 2, Row 4: space-*_electrodes.tsv
\node[widecontentbox, below=0.6cm of box5.south west, anchor=north west] (box7) {
    \textbf{\normalsize\ttfamily *\_space-*\_electrodes.tsv}\\[2pt]
    {\footnotesize\textcolor{gray}{Electrodes (stereotaxic)}}\\[4pt]
    {\tiny\ttfamily
    \begin{tabular}{@{}llllllll@{}}
    \toprule
    name & probe\_name & hemisphere & x & y & z & impedance & location \\
    \midrule
    e001 & probe01 & L & -2.50 & 1.50 & -4.00 & 1.2 & RegionA \\
    e002 & probe01 & L & -2.50 & 1.50 & -4.02 & 1.1 & RegionA \\
    e003 & probe01 & L & -2.50 & 1.50 & -4.05 & 1.3 & RegionA \\
    \bottomrule
    \end{tabular}
    }
};
\node[labelstyle] at ([xshift=-0.4cm, yshift=-0.4cm]box7.north east) {6};

% Column 3, Row 1: probes.json
\node[narrowcontentbox, right=0.2cm of box1.north east, anchor=north west] (box2) {
    \textbf{\normalsize\ttfamily *\_probes.json}\\[2pt]
    {\footnotesize\textcolor{gray}{ProbeInterface}}\\[4pt]
    {\tiny\ttfamily
    \{\\
    ~~"model": \{\\
    ~~~~"Description": "Probe model, referencing ProbeInterface",\\
    ~~~~"Levels": \{\\
    ~~~~~~"ModelA": \{"TermURL": "https://raw.github..."\},\\
    ~~~~~~"ModelB": \{"TermURL": "bids::probes/custom.json"\}\\
    ~~~~\}\\
    ~~\},\\
    ~~"type": \{\\
    ~~~~"Levels": \{\\
    ~~~~~~"silicon-probe": "Multi-electrode silicon probe",\\
    ~~~~~~"tetrode": "Bundle of four twisted wires"\\
    ~~~~\}\\
    ~~\},\\
    ~~"coordinate\_reference\_point": \{\\
    ~~~~"Levels": \{"tip": "Probe tip"\}\\
    ~~\},\\
    ~~"anatomical\_reference\_point": \{\\
    ~~~~"Levels": \{"Bregma": "Coronal/sagittal junction"\}\\
    ~~\}\\
    \}
    }
};
\node[labelstyle] at ([xshift=-0.4cm, yshift=-0.4cm]box2.north east) {8};

% Column 3, Row 2: coordsystem.json
\node[narrowcontentbox, below=0.6cm of box2.south west, anchor=north west] (box6) {
    \textbf{\normalsize\ttfamily *\_space-*\_coordsystem.json}\\[2pt]
    {\footnotesize\textcolor{gray}{Coordinate System}}\\[4pt]
    {\tiny\ttfamily
    \{\\
    ~~"MicroephysCoordinateSystem": "StereoTaxic",\\
    ~~"MicroephysCoordinateUnits": "mm",\\
    ~~"MicroephysCoordinateSystemDescription":\\
    ~~~"Stereotaxic coords rel. to Bregma. Right-handed:\\
    ~~~AP (anterior-posterior, + = anterior),\\
    ~~~ML (medial-lateral, + = right),\\
    ~~~DV (dorsal-ventral, + = ventral). Origin at Bregma.",\\
    ~~"IntendedFor": "sub-01/ses-001/ecephys/sub-01\_ses-001\_task-\\
    ~~~discrimination\_ecephys.nix"\\
    \}
    }
};
\node[labelstyle] at ([xshift=-0.4cm, yshift=-0.4cm]box6.north east) {7};

% Background
\begin{scope}[on background layer]
\fill[white] (current bounding box.south west) rectangle (current bounding box.north east);
\end{scope}

\end{tikzpicture}
\end{document}
